\subsection{Outcome modelling}

\subsubsection{Outcome modelling overview}

Detailed methods and code used for modelling these outcomes are available \cite{github2}, with methods described as an online book \cite{github3}. The outcome model is available as a PyPI package for Python \cite{pypi}.

We used modified Rankin Scale (mRS) at 6 months as a measure of outcome. mRS is the most commonly used instrument to describe post-stroke functional outcome \cite{quinn_functional_2009}, describing independence of living from a scale of 0 (no disability) through to 5 (severe disability requiring constant nursing attention), with death assigned an mRS of 6. A commonly used surrogate for independent living is  mRS 0-2. Health utility values for each mRS level were taken from Wang \textit{et al.} \cite{wang_utility-weighted_2020}. The mean mRS score, mean utility and proportion of patients with mRS 0-2 in a given mRS distribution can be compared between scenarios.

We calculated the patients mRS outcome distribution based on time to treatment for three patient-treatment cohorts: nLVO treated with IVT; LVO treated with IVT alone; and LVO treated with IVT and MT. For each patient-treatment cohort we calculated an mRS distribution for treatment at any given time by interpolating between the mRS distribution for treatment given at \emph{t=0} (time of stroke onset) and the mRS distribution for treatment given at \emph{t=No Effect} (time of no effect of treatment), assuming that log odds fall linearly over time \cite{emberson_effect_2014, fransen_time_2016}. Further details on how these \emph{t=0} and \emph{t=No Effect} mRS distributions were derived are given in the supplementary material.

The time to no effect was 6.3 hours for IVT \cite{emberson_effect_2014} and 8.0 hours for MT \cite{ fransen_time_2016}. Our model did not include selection of patients who may still benefit from treatment beyond these durations through the use of perfusion scanning. This number is small for IVT, but is more substantial for MT – approximately 2500 per annum in England. 

\begin{minipage}{1.0\textwidth}  % Define the width of the minipage
\begin{longtable}{p{1.2cm} p{13cm}}
\caption{Description of modified Rankin Scale (mRS)categories}\label{tab:mrs}\\
\toprule
mRS & Description \\
\midrule
0 & No symptoms. \\
1 & No significant disability. Able to carry out all usual activities,
despite some symptoms. \\
2 & Slight disability. Able to look after own affairs without
assistance, but unable to carry out all previous activities. \\
3 & Moderate disability. Requires some help, but able to walk
unassisted. \\
4 & Moderately severe disability. Unable to attend to own bodily needs
without assistance, and unable to walk unassisted. \\
5 & Severe disability. Requires constant nursing care and attention,
bedridden, incontinent. \\
6 & Dead. \\
\bottomrule
\end{longtable}
\end{minipage} 


