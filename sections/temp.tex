\section{MSUs and thrombectomy}

The BEST-MSU substudy \cite{czap_abstract_2022} was a subset of the best MSU study, for tPA-eligible stroke patients with LVOs on CT and/or CTA. The study appeared to show MSUs have little effect on time to thrombectomy. A total of 295 patients were included, 169 in the MSU group and 126 in the EMS group. 92\% MSU vs 87\% EMS LVO patients received tPA, and 78\% vs 85\% went on to have EVT. MSU LVO patients had faster tPA bolus from symptom onset (65 min vs 96 min, p<0.001), however the two groups had similar onset to groin puncture (169 min vs 162 min, p=0.77). From the BEST study \cite{grotta_prospective_2021}, the median (and IQR) time from 911 alert to  thrombectomy was 141 (116–171) and 132 minutes (114–160) for MSU and EMS, and  time from ED door to thrombectomy was 76 (53–105) and 94 (72–124).


The Melbourne MSU study \cite{menezes_abstract_2023} found MSUs improved thrombectomy use and speed during, but not before, the COVID pandemic. A total of 402 patients (112 MSU) were included. Pre-pandemic, no reduction in dispatch to arterial access time was seen for MSU patients within an EVT centre catchment (median 11 min slower, p=0.38). However, a significant time saving was observed during the pandemic (median 29 min faster, p<0.001, p=0.0065). MSU care reduced hospital arrival to arterial access time by median 19 min pre-pandemic vs 40 min during the pandemic, p<0.001).

In Sydney the MSU did not improve time or rate of thrombectomy \cite{haliem_abstract_2023} but the MSU dispatch missed 40\% of those patients that would go on to receive thrombectomy. An odd finding there was also the ambo dispatch process was less likely to class severe strokes as stroke, compared to mild stroke. A total of n=618 patients were included with baseline NIHSS 16 (IQR 10-20). Of these, only 62\% (95\% CI 58-66) were initially dispatched as suspected stroke, with the most common non-stroke diagnoses being “Unconscious/Fainting” (19.2\%) and “Falls” (6.9\%). Those with a higher baseline severity (NIHSS $\ge$ 10) were less likely to be classified as stroke than those with lower severity (59\% vs 76\%, p<0.001), while no difference was found between metropolitan and rural patients (p=0.066). Overall, no significant time differences were found between stroke and non-stroke dispatches for ambulance dispatch to arterial access (median 208 vs 216 min, p=0.593) or hospital arrival to arterial access (median 42 vs 42 min, p=0.851). However, only 32 patients were treated on the MSU, which commenced operation November 2017 (2007-2021 was used for all data analysis). 


A review of available evidence suggests there is an evidence gap in MSUs and thrombectomy \cite{navi_mobile_2022}.